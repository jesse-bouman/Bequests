\chapter{Temperature dependence of fitted TQ parameters\label{app1}}
The fitted parameters are shown in the graphs below. Every row represents a bell curve, the first column showing the prefactor of the curve, $N_i$, the second showing the peak energy, $\mu_i$ and the third indicating the width of the peak, $\sigma_i$. In the algorithm, the fitted parameters are given bounds in which they can be fitted freely. This is done in order to prevent the algorithm of "switching" functions. These bounds are set to follow the temperature dependence of the parameters. These bounds are shown in green, the fitted parameters in blue. It can be seen that at temperatures lower than $100$ \degree{}C, none of the parameters hit the floor or the ceiling set by the bounds, so the found relation is physically relevant.
\begin{figure}
    \centering
    \includegraphics[scale=0.18]{PlotFitParameters.png}
\end{figure}

\chapter{Determination of zero-phonon energies for the ${}^4T_2$ band\label{app2}}
\begin{figure}
\centering
\begin{subfigure}{0.7\textwidth{}}
    \centering
    \includegraphics[width=\textwidth{}]{emssionEzoom0.png}
    \caption{$x=0$}
\end{subfigure}
\begin{subfigure}{0.7\textwidth{}}
    \centering
    \includegraphics[width=\textwidth{}]{emssionEzoom1.png}
    \caption{$x=1$}
\end{subfigure}
\caption{}
\end{figure}

\begin{figure}
\centering
\begin{subfigure}{0.7\textwidth{}}
    \centering
    \includegraphics[width=\textwidth{}]{emssionEzoom2.png}
    \caption{$x=2$}
\end{subfigure}
\begin{subfigure}{0.7\textwidth{}}
    \centering
    \includegraphics[width=\textwidth{}]{emssionEzoom4.png}
    \caption{$x=4$}
\end{subfigure}
\begin{subfigure}{0.7\textwidth{}}
\centering
\includegraphics[width=\textwidth{}]{emssionEzoom5.png}
\caption{$x=5$}
\end{subfigure}
\caption{}
\end{figure}
